\section{Linear regression}

\begin{tcolorbox}[width=\linewidth, sharp corners=all, colback=white!95!black]
We are interested in the basic linear regression model where given observations $(x_i, y_i)_{1\leq i \leq n}$ we want to build the model $$Y = \alpha + \beta X + \varepsilon.$$

Explain the underlying hypotheses in the model and derive estimators for the coefficients.

\end{tcolorbox}

The underlying assumptions of the linear model are the following :

\begin{itemize}
    \item No perfect collinearity between the explanatory variables, otherwise the parameter $\beta$ is not identifiable.
    \item Homoscedasticity, or constant variance, which can be tested on the residuals. If there is heteroscedasticity, the Gauss-Markov theorem doesn't apply, thus the estimators derived are not the Best Linear Unbiased Estimators (BLUE). It can be corrected thanks to a weighted least squares approach, or a logarithmization of the data.
    \item Independence of errors. Generalized least squares can handle correlated errors.
    \item Exogeneity 
\end{itemize}

We have to minimize the Euclidean distance between the predicted values by the model for $Y$, $\hat{Y}$ and the actual values. This can be written as :
$$(\hat{\alpha}, \hat{\beta}) \in \operatorname{argmin}_{(\alpha, \beta) \in \mathbb{R}^2} \displaystyle \sum_{i=1}^{n} (y_i - \hat{y_i})^2$$

Replacing with the model, we have to find parameters that minimize $f(\alpha, \beta) = \displaystyle \sum_{i=1}^{n} (y_i - (\alpha + \beta x_i))^2.$ We will write the first order conditions and check by computing the Hessian that we are actually looking at a minimum.

$$\left\{\begin{array}{ll}
        \dfrac{\partial f}{\partial \alpha}(\alpha, \beta) &= 0 \\
        \dfrac{\partial f}{\partial \beta}(\alpha, \beta) &= 0
    \end{array}\right. \Leftrightarrow 
\left\{\begin{array}{ll}
        -2\sum_i (y_i - \alpha - \beta x_i) &= 0 \\
        -2\sum_i x_i (y_i - \alpha - \beta x_i) &= 0
    \end{array}\right.$$

The first line give $\hat{\alpha} = \overline{y} - \hat{\beta} \overline{x}$ while the second line can be written as the following : 
\begin{equation*}
\begin{aligned}
    \sum_i x_i (y_i - \hat{\alpha} - \hat{\beta} x_i) &= \sum_i y_i x_i - (\overline{y} - \hat{\beta} \overline{x})x_i - \hat{\beta} x_i^2\\
    &= \sum_i y_i - \overline{y} + \hat{\beta} (\overline{x} - x_i).
\end{aligned}
\end{equation*}

Thus 
\begin{equation*}
\begin{aligned}
    &\sum_i x_i (y_i - \hat{\alpha} - \hat{\beta} x_i) = 0 &\Leftrightarrow \sum_i y_i - \overline{y} + \hat{\beta} (\overline{x} - x_i) = 0\\
    &\Leftrightarrow \hat{\beta} = \dfrac{\sum_i y_i - \overline{y}}{\sum_i x_i - \overline{x}}\\
    &\Leftrightarrow \hat{\beta} = \dfrac{\sum_i (y_i - \overline{y})(x_i-\overline{x})}{\sum_i (x_i - \overline{x})^2}\\
    &\Leftrightarrow \boxed{\hat{\beta} = \dfrac{\widehat{\operatorname{Cov}}(X, Y)}{\widehat{\operatorname{Var}}(X)}.}
\end{aligned}
\end{equation*}



