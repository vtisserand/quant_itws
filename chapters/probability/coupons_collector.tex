\section{The coupons collector}

\begin{tcolorbox}[width=\linewidth, sharp corners=all, colback=white!95!black]
A chocolate company launches a marketing campaign: for each chocolate bar you buy, you get one collectible card out of a set of $n$ possible cards. We can assume the card are uniformly distributed among the chocolate bars.\newline
How many chocolate bars should you buy to complete the collection?
\end{tcolorbox}

Well, at least $n$, even if we are very lucky.\newline
The first bar we open will yield to a new card. For the second bar, we have a probability $\dfrac{1}{n}$ to get the same card we already have, thus $\dfrac{n-1}{n}$ to get a new card. This follows a geometric law: the expectation for such an event is $\dfrac{n}{n-1}$. And so on, decreasing the probability for each new card we acquire.\newline
The total expectancy will be the sum of all of these individual processes: 
$$\mathbb{E}\left[ N \right] = \displaystyle \sum_{k=0}^{n-1} \dfrac{n}{n-k},$$
with N the random variable that counts the number of chocolate bars eaten to get the full collection.\\

We realize that we are actually dealing with the harmonic sum $H_n = \sum_{k=1}^{n} \frac1{k}$, which can be squeezed between two integrals to get the equivalent: $H_n \sim_{n \to +\infty} \log(n).$\newline
Thus $N \sim_{n \to +\infty} n\log(n).$\\

To give a confidence interval around the number of chocolate bars we should buy, let's pull up some concentration inequalities.\newline
We still deal with the sum of independent geometric variables so the variance is easy to compute:

\begin{align*}
    \operatorname{Var}[N] &= \displaystyle \sum_{k=1}^n \operatorname{Var}[N_i]\\
    &= \displaystyle \sum_{k=1}^n \left(1 - \dfrac{n-k+1}{n}\right)\left(\dfrac{n}{n-k+1}\right)^2\\
    &= n\displaystyle \sum_{k=1}^n \dfrac{k-1}{(n-k+1)^2} \sim_{n \to +\infty} n \dfrac{\pi^2}{6}. 
\end{align*}

Applying Chebychev's inequality, we get: $$\mathbb{P}(\lvert \mathbb{E}[N] - N \rvert \geq k \sigma) \leq \dfrac{1}{k^2}.$$

Some other inequalities could be used to raffinate this result: Chernoff bounds, Vysochanskij–Petunin inequality, etc.


