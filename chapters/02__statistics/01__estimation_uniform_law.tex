\section{Estimating the support of an uniform law}

\begin{tcolorbox}[width=\linewidth, sharp corners=all, colback=white!95!black]
Suppose that we have $x_1,\dots,x_n$ observations from an uniform law $X \sim \mathcal{U}[0,\theta]$, where $\theta$ is an unknown parameter that we want to estimate. Give at least two estimators for $\theta$ and compare them.
\end{tcolorbox}

\paragraph*{Method of moments:}\\

Having a look at the first order moment, it appears that $\mathbb{E}[X] = \theta/2$. Taking the empirical counter-party of this theoretical quantity, we have $\boxed{\hat{\theta}^{\text{MM}} = \dfrac{2}{n}\sum\limits_{i=1}^n x_i.}$\newline
By applying the strong law of large numbers and the continuous mapping theorem, $\hat{\theta}^{\text{MM}} \xrightarrow{a.s.} \theta$. Thus \boxed{\text{this estimator is consistent}}.\newline
We want asymptotic results on the convergence of this estimator. Before using the CLT, we have to check for the existence of a second-order moment.

\begin{align*}
    \mathbb{E}[X^2] &= \displaystyle \int_{\mathbb{R}} x^2 f(x) \, \mathrm{d}x \\
    &= \displaystyle \int_{0}^{\theta} x^2 \dfrac{1}{\theta} \, \mathrm{d}x \\
    &= \left[ \dfrac{1}{3\theta}x^3 \right]^\theta_0 \\
    &= \dfrac{\theta^2}{3} < +\infty
\end{align*}

Thus, we have $\mathbb{V}[X] = \mathbb{E}[X^2] - \mathbb{E}[X]^2 = \dfrac{\theta^2}{12}$. So, $\mathbb{V}[2X_1]  = \dfrac{\theta^2}{3}$.\newline

By applying the central limit theorem, we have : $$\sqrt{n}(\hat{\theta}^{\text{MM}} - \theta) \xrightarrow{(d)} \mathcal{N}\left(0, \dfrac{\theta^2}{3}\right).$$

We have to evaluate the risk of this estimator, that we write as the sum of the squared bias and the variance : $$\text{MSE}(\hat{\theta}^{\text{MM}}) = \mathbb{E}[(\hat{\theta}^{\text{MM}} - \theta)^2] = \mathbb{E}[(\hat{\theta}^{\text{MM}} - \mathbb{E}[\hat{\theta}^{\text{MM}}])^2] + \mathbb{E}[\hat{\theta}^{\text{MM}} - \theta]^2 = \mathbb{V}[\hat{\theta}^{\text{MM}}] + (\mathbb{E}[\hat{\theta}^{\text{MM}}] - \theta)^2.$$

We have $\mathbb{E}[\hat{\theta}^{\text{MM}}] = 0$ and $\mathbb{V}[\hat{\theta}^{\text{MM}}] = \dfrac{1}{n^2}n\mathbb{V}[2X_1] = \dfrac{\theta^2}{3n}.$\newline

Thus, $$\text{MSE}(\hat{\theta}^{\text{MM}}) = \dfrac{\theta^2}{3n}.$$


\paragraph*{Maximum likelihood:}

Let's write the likelihood of this model:

\begin{align*}
    L((X_1,\dots,X_n), \theta) &= \displaystyle \prod_{i=1}^{n} f_X(X_i) \\
    &= \displaystyle \prod_{i=1}^{n} \dfrac{1}{\theta} \mathbbm{1}_{[0, \theta]}(X_i)\\
    &= \dfrac{1}{\theta^n} \prod_{i=1}^{n} \mathbbm{1}_{[0, \theta]}(X_i).
\end{align*}

And this function is maximized by choosing the smallest $\theta$ such that all of the $X_i$ lie in $[0, \theta]$, that is $\hat{\theta}^{\text{MLE}} = \max_{1\leq i \leq n} X_i.$\newline

To check the consistency of this estimator, we will have a look at its convergence (in probability). Let $\theta \in \Theta$ and $\varepsilon >0$ :

\begin{align*}
    \mathbb{P}_\theta(\lvert\hat{\theta}^{\text{MLE}} - \theta\rvert \ge \varepsilon) &= \mathbb{P}_\theta(\hat{\theta}^{\text{MLE}} \ge \theta + \varepsilon) + \mathbb{P}_\theta(\hat{\theta}^{\text{MLE}} \le \theta - \varepsilon)\\
    &= 0 + \mathbb{P}_\theta(\max_{1\leq i \leq n} X_i \le \theta - \varepsilon)\\
    &= \prod_{i=1}^{n} \mathbb{P}_\theta(X_i \le \theta - \varepsilon)\\
    &= \left(1 - \dfrac{\varepsilon}{\theta} \right)^n \underset{n\to +\infty}{\longrightarrow} 0.
\end{align*}


Thus, $\hat{\theta}^{\text{MLE}} \xrightarrow{\mathbb{P}} \theta$ : this estimator is consistent.\newline

In order to estimate the risk of this estimator, we have to look at the law that the maximum of $n$ independent uniform laws follows. This is done by looking at the cumulative distribution function. Let $x \in [0,\theta] :$

\begin{align*}
    \mathbb{P}_\theta(X_{(n)} \le x) &= \mathbb{P}_\theta\left(\bigcap_{i=1}^{n} {X_i \le x}\right)\\
    &= \prod_{i=1}^{n} \mathbb{P}_\theta(X_i \le x)\\
    &= \left(\dfrac{x}{\theta}\right)^n.
\end{align*}

Thus, $$F_{X_{(n)}} = \begin{cases}
        0 & \text{if } x < 0\\
        \left(\dfrac{x}{\theta}\right)^n & \text{if } 0 \le x \le \theta\\
        1 & \text{if } x>\theta
        \end{cases}$$

This cdf as smooth as we need to take its derivative: that will be the density we were looking for:

$$f_{X_{(n)}}(x) = n \dfrac{x^{n-1}}{\theta^n} \mathbbm{1}_{[0, \theta]}(x)$$

Let's compute the bias and the variance.\newline

$$\mathbb{E}[\hat{\theta}^{\text{MLE}}] = \displaystyle \int_{\mathbb{R}} x f_{X_{(n)}}(x) \, \mathrm{d}x = \displaystyle \int_{0}^{\theta} \dfrac{n}{\theta^n} x^n \, \mathrm{d}x = \dfrac{n}{\theta^n} \left[ \dfrac{x^{n+1}}{n+1}\right]^\theta_0 = \dfrac{n}{n+1}\theta.$$

Then, the bias is : $B(\hat{\theta}^{\text{MLE}}) = \dfrac{n}{n+1}\theta - \theta = -\dfrac{1}{n+1}\theta \ne 0.$ We can introduce a corrected estimator that we will consider too : $\hat{\theta}_{\text{corr}}^{\text{MLE}} = \dfrac{n+1}{n} \hat{\theta}^{\text{MLE}}$, such that $\mathbb{E}[\hat{\theta}_{\text{corr}}^{\text{MLE}}] = \theta$:  an unbiased estimator.\newline

Then, we have $$\mathbb{E}[(\hat{\theta}^{\text{MLE}})^2] = \displaystyle \int_{\mathbb{R}} x^2 f_{X_{(n)}}(x) \, \mathrm{d}x = \displaystyle \int_{0}^{\theta} \dfrac{n}{\theta^n} x^{n+1} \, \mathrm{d}x = \dfrac{n}{\theta^n} \left[ \dfrac{x^{n+2}}{n+2}\right]^\theta_0 = \dfrac{n}{n+2}\theta^2.$$ And $$\text{MSE}(\hat{\theta}^{\text{MLE}}) = \mathbb{E}[(\hat{\theta}^{\text{MLE}} - \theta)^2] = \mathbb{E}[(\hat{\theta}^{\text{MLE}})^2] -2 \theta \mathbb{E}[\hat{\theta}^{\text{MLE}}] + \theta^2$$

Thus, $$\text{MSE}(\hat{\theta}^{\text{MLE}}) = \dfrac{n}{n+2}\theta^2 - 2\dfrac{n}{n+1}\theta^2 + \theta^2 = \dfrac{2\theta^2}{(n+1)(n+2)}.$$

And $$\text{MSE}(\hat{\theta}_{\text{corr}}^{\text{MLE}}) = \left(\dfrac{n+1}{n}\right)^2 \mathbb{E}[(\hat{\theta}^{\text{MLE}})^2] - 2\dfrac{n+1}{n}\theta \mathbb{E}[(\hat{\theta}^{\text{MLE}} + \theta^2 = \dfrac{\theta^2}{n(n+1)}.$$

\paragraph*{Maximum a posteriori:}

We write the likelihood of the model in terms of $\theta$ :
$$L((X_1,\dots,X_n), \theta) = \dfrac{1}{\theta^n} \prod_{i=1}^{n} \mathbbm{1}_{[0, \theta]}(X_i) = \dfrac{1}{\theta^n} \mathbbm{1}_{[X_{(n)}, =\infty[}(\theta).$$

\textbf{Remark :} the set such that $L(., \theta)>0$ is $[0, \theta]$ : it depends on $\theta$, thus the model is not regular. Keep that in mind when dealing with Fisher information for instance.\newline

\begin{enumerate}
    \item \textbf{Flat prior:}\newline

We apply the definition for a Bayesian estimator with a prior density $\pi_0$ :

\begin{align*}
    \hat{\theta}^{\text{B}} &= \dfrac{\displaystyle \int_{\Theta} \theta L(x, \theta) \pi_0(\theta) \, \mathrm{d}\lambda(\theta)}{\displaystyle \int_{\Theta} L(x, \theta) \pi_0(\theta) \, \mathrm{d}\lambda(\theta)}\\
    &= \dfrac{\displaystyle \int_{X_{(n)}}^{+\infty} \theta^{-n+1} \, \mathrm{d}\theta}{\displaystyle \int_{X_{(n)}}^{+\infty} \theta^{-n} \, \mathrm{d}\theta}\\
    &= \dfrac{n-1}{n-2}X_{(n)}.
\end{align*}

Bias and MSE are not computed there for sanity reasons.

    \item \textbf{Jeffreys prior:}\newline

The density function of this prior is proportional to the squareroot of the determinant of the Fisher information matrix.\newline

Thus we need to compute this quantity for this model, with n observations (as it is not regular, $I_n \ne n I_1$) :

$$I_n(\theta) = \mathbb{E}\left[\dfrac{\partial \log L_n(\theta)}{\partial \theta}^2\right]$$

We have $I_n(\theta) = \mathbb{E}[(-n/\theta)^2] = \dfrac{n^2}{\theta^2}$\newline

(If we had taken the expectancy of the second-order derivative of the log-likelihood, we would not have had the same result has the model is not regular.)\newline

This gives us the noninformative prior (Jeffreys) : $\pi_0(\theta) \propto \theta^{-1}.$


\end{enumerate}




