\section{Dupire formula for local volatility}

\begin{tcolorbox}[width=\linewidth, sharp corners=all, colback=white!95!black]
Explain the motivation behind local volatility and prove Dupire formula.
\end{tcolorbox}

The Black-Scholes model is really convenient as it has few parameters, yields closed-form vanilla prices and is widely known. However, the assumption of constant volatility doesn't match the observed market prices.\newline Indeed, looking at the implied volatility surface we observe smile / skew / smirk accross all asset classes. In equities, the volatility smile appeared after the 1987 crisis and reflected this premium buyers were ready to pay to hedge against the downside risk.\\
In order to build new pricing models that reflected this stylized fact, a maturity and strike dependent volatility was introduced: going from a constant $\sigma$ accros all instruments to $\sigma(t, S_t)$.\\

What is now known as the Dupire formula is the following expression for such a volatility function: $$\sigma(t, S_t) = \dfrac{\dfrac{\partial C}{\partial T} + (r-q)K \dfrac{\partial C}{\partial{K}} + qC}{\dfrac{1}{2}K^2 \dfrac{\partial^2{C}}{\partial{K^2}}}.$$

We assume the underlying follows a Geometric brownian motion dynamic: $dS_t = \mu S_t dt + \sigma^2 S_t dW_t$, with $\mu = r-q$. We also introduce the discount factor between time $t$ and maturity $T$: $D(t,T) = \exp \left( -\displaystyle \int_t^T r(s) ds \right)$.\newline The call option price can then be written as $C(K, T) = D(t, T)\mathbb{E}_{\mathbb{Q}\left[(S_T - K)^{+}\right]}$.\\

We are interested in the probability density of the underlying at maturity: $p(S, t)$. Its variations are governed by the \textbf{Fokker-Planck equation}: $$\dfrac{\partial}{\partial t}p(S,t) = -\dfrac{\partial}{\partial S}(\mu Sp(S,t)) + \dfrac{1}{2} \dfrac{\partial^2}{\partial \sigma^2}(\sigma^2 S^2 p(S,t)).$$\newline

Let's compute the theta of a call option: $$\dfrac{\partial C}{\partial T} = \dfrac{\partial D(t,T)}{\partial T} \displaystyle \int_K^{+\infty} (S - K)p(S, T-t)dS + D(t,T)\int_K^{+\infty} (S - K)\dfrac{\partial p(S, T-t)}{\partial T}dS.$$ Plugging in we get:

\begin{align*}
    \Theta + rC &= D(t,T)\displaystyle \int_{K}^{+\infty} (S-K) \left[-\dfrac{\partial}{\partial S}(\mu Sp(S,t)) + \dfrac{1}{2} \dfrac{\partial^2}{\partial \sigma^2}(\sigma^2 S^2 p(S,t))\right]\\
    &= D(t,T)\left(-\mu I_1 + \dfrac{1}{2}I_2\right).
\end{align*}

We consider the first and second order derivatives with regards to the strike. Is is known that they respectively are equal to the cumulative distribution function above the strike and the probability density at maturity (the latter being the \textbf{Breeden-Litzenberger formula}).\newline To know what quantities we should further consider, we apply integration by parts to $I_1$ and $I_2$, with the goal to get rid of integrands and fuzzy terms.\\

\begin{align*}
    I_1 &= \displaystyle \int_{K}^{+\infty} (S-K) \dfrac{\partial}{\partial S}(Sp(S,t))\\
    &= \left[(S - K)p(S,t)\right]_{S=K}^{S=+\infty} - \displaystyle \int_K^{+\infty} S p(S,t)dS\\
    &= - \displaystyle \int_K^{+\infty} S p(S,t)dS.
\end{align*}

To explicit this last line, let's rewrite the call price as C = Se^{-qT} \mathcal(d_1) - K e^{-rT} \mathcal(d_2)$ such that $\displaystyle \int_K^{+\infty} S p(S,t)dS = \dfrac{1}{D(t,T)}\left(C - K \dfrac{\partial C}{\partial K}\right)$.\\

Then, 

\begin{align*}
    I_2 &= \displaystyle \int_{K}^{+\infty} (S-K) \dfrac{\partial^2}{\partial \sigma^2}(\sigma^2 S^2 p(S,t))dS\\
    &= \left[(S - K)\dfrac{\partial}{\partial \sigma}(\sigma^2 S^2 p(S,t))\right]_{S=K}^{S=+\infty} - \displaystyle \dfrac{\partial}{\partial \sigma}(\sigma^2 S^2 p(S,t))dS\\
    &= - \left[(\sigma^2 S^2 p(S,t)\right]_{S=K}^{S=+\infty}\\
    &= \sigma^2 K^2 p(S,t)\\
    &= \sigma^2 K^2 \dfrac{1}{D(t,T)} \dfrac{\partial^2 C}{\partial K^2}.
\end{align*}

Going back to the theta derivation, we have $$\dfrac{\partial C}{\partial T} + rC = C - K\dfrac{\partial C}{\partial K} + \sigma^2 K^2 \dfrac{\partial^2 C}{\partial K^2}.$$
 
Rearranging the terms, we get the Dupire formula.\\

There also exist a probabilistic derivation of this formula, applying Itô to the payoff $(S_T - K)^{+}$ and taking the expectation.











