\section{Rating a signal}

\begin{tcolorbox}[width=\linewidth, sharp corners=all, colback=white!95!black]
We are backtesting a trading signal. While there are already a lot of caveats in this first step, we are interested here in "grading" and comparing this signal quality to others. How would you do so?
\end{tcolorbox}

\paragraph*{Performance metrics}



\begin{figure}[H]
    \includegraphics[width=0.85\textwidth]{include/img/banks_statarb_equity_curve.png}
    \centering
    \caption{Equity curve of an volume autocorrelation strategy on major bank stocks traded on the NYSE, based on an article from \href{https://x.com/systematicls/status/1802666506125558115}{@systematicls}. See \href{https://github.com/vtisserand/quant_itws/tree/main/code/snippets/bank_statarb_signal.py}{code snippet}.}
    \label{fig:banks_equity_curve}
\end{figure}

\paragraph*{Minimum track record length}

A recently onboarded PM claims to have a strategy yielding an annualized Sharpe of $2.0$, from annualized $10\%$ return and annualized volatility $5\%$. After a year of trading, they are down $3\%$ with a volatility of $9\%$. Can they be held accountable for such a performance?

To conclude on the signal the PM claims to be trading, we can build confidence intervals on the actual Sharpe ratio given the number of days of trading.

With classic estimators for the mean and variance of the series, $\hat{\mu}$ and $\hat{\sigma}^2$, the estimator for the Sharpe ratio follows: $\widehat{SR} = \dfrac{\hat{\mu} - R_f}{\hat{\sigma}} = g(\hat{\mu}, \hat{\sigma}^2)$.\newline We can derive asymptotic properties on this estimator with the Central Limit Theorem: assuming i.i.d. returns, we have: \[\sqrt{T}(\widehat{SR} - SR) \stackrel{(d)}{\underset{T \to \infty}{\longrightarrow}} \mathcal{N}(0, \operatorname{Var}(SR)).\]

We just need to explicit the variance here: we are considering a transformation of the column estimator $\hat{\theta} = (\hat{\mu}, \hat{\sigma}^2)^{T}$. $\sqrt{T}(\hat{\theta} - \theta) \overset{(d)}{\longrightarrow} \mathcal{N}(0, \operatorname{Var}(\theta))$, where $\operatorname{Var}(\theta) = \begin{pmatrix} \sigma^2 & 0\\ 0 & 2\sigma^4 \end {pmatrix}.$ \newline This is now an application of the \textit{delta method}: \[\sqrt{T}(\widehat{SR} - SR) \overset{(d)}{\longrightarrow} \mathcal{N}\left(0, \dfrac{\partial g}{\partial \theta}^{T} \operatorname{Var}(\theta) \dfrac{\partial g}{\partial \theta}\right).\]

$\dfrac{\partial g}{\partial \theta}^{T} = \left(\dfrac{\partial g}{\partial \mu}, \dfrac{\partial g}{\partial \sigma^2}\right) = \left(1/\sigma, -(\mu - R_f) / (2\sigma^3)\right).$

Thus, $\operatorname{Var}(SR) = 1 + \dfrac{(\mu - R_f)^2}{2\sigma^2} = 1 + \frac1{2}SR^2.$

We can now compute standard errors (SEs) for the Sharpe ratio as $SE(\widehat{SR}) = \sqrt{\frac1{T}(1+\frac1{2} SR^2)}$, allowing the $95\%$ confidence bounds: $\widehat{SR} \pm 1.96 \times \sqrt{\frac1{T}(1+\frac1{2} SR^2)}$. The higher the Sharpe, the noisier the estimation. Overall the confidence enveloppe decreases as $T^{-1/2}$.\newline For small Sharpes ($< 1$), the contribution of the erros in the mean estimator matter the most, while for higher Sharpes the variance matters more, see \cite{lo2002statistics}.