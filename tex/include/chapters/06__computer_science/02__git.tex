\section{Using \texttt{git}}

\begin{tcolorbox}[width=\linewidth, sharp corners=all, colback=white!95!black]
    \pythoninline{git} is the go-to version control system to track changes in files and collaborate with other programmers.
    Below is a list of the basic survival commands, as well as some more in-depth routines -- always refer to a \href{https://git-scm.com/docs}{good documentation} though, where all the possible arguments are detailed.
\end{tcolorbox}

\subsection*{Basics}

Let's say it is your first day at an investment bank as a quant. Soon enough you're going to get developper access to a bunch of libraries (used for pricing, monitoring, or whatever) that you are going to contribute to.\newline
\texttt{git clone}\newline
is going to be your first command to get your local version of the remote repository. If you were to create a new repository, you'd use \texttt{git init}.

Then you're going to add features to the current library, for instance implement this fancy stochastic volatility model you've heard about. There is the \texttt{main} or \texttt{master} branch that everyone takes as reference, and you'll want to create your changes on a parallel version to not interfere. Instead of using \texttt{git branch} to create your own branch, you can directly go with:\newline
\texttt{git checkout -b \$BRANCH_NAME}\newline
Now you can start adding / modifying files. Switch between branches with \texttt{git switch} and toggle between the current and the last seen with \texttt{git checkout -}.

A nice routine to add, commit and push modifications is:\newline
\texttt{git add .}\newline
\texttt{git commit -m "\$COMMIT_MESSAGE"}\newline
\texttt{git push}

A good commit meassage should look like \texttt{feat: fast pricing for heston model}, \texttt{fix(params): update default buffer for edge case} or \texttt{docs: added equations in the docstring}, specifying the type of the modification and even (sometimes) an additional scope in parentheses.

To stay on top of the modifications of your colleagues and avoid conflicts, keep regularly pulling the latest changes:\newline
\texttt{git pull origin master}\newline
which is a shortcut for \texttt{git fetch origin} and \texttt{git merge}.

\texttt{git log} is useful to show an history of the commits, with messages and commit references. A better log can be seen with:\newline
\texttt{git log --pretty=oneline --abbrev-commit --graph --decorate --all}

\subsection*{Intermediate}

You are now well established in the \texttt{git} world. Several features are being developped simultaneously on different branches. There is an emergency and you need to switch to the branch \texttt{feature1} while you are developping on \texttt{feature2}. However you cannot add your changes to the staging area as they are not quite ready yet. You can \texttt{git stash} the current changes in a dirty working directory to keep the current state. The stash works like a stack, you can visualize it with \texttt{git stash list}, \texttt{pop} it to apply the oldest change and delete it from the stash, or \texttt{apply} it while keeping it in the ditry working directory. Now you can switch to the urgent feature and come back to the current and reapply the modifications.

\texttt{git rebase} allows to play with branches in such a way:

\begin{center}
\begin{verbatim}
      A---B---C   feature                      A'--B'--C'   feature
     /                      -->               /          
D---E---F---G   master           D---E---F---G   master   
\end{verbatim}
\end{center}

\texttt{git cherry-pick <commit>} applies the changes of a commit.


\subsection*{Advanced}