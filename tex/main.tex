\documentclass[english, a4paper, 12pt]{report}

% Overall document aspect
\usepackage[utf8]{inputenc}
\usepackage[english]{babel}
\usepackage[toc,page]{appendix}
\usepackage[skip=0.75\baselineskip plus 2pt]{parskip}
\usepackage{geometry}
\geometry{vmargin=2.5cm}
\usepackage{changepage}
\usepackage{adjustbox}

% For specific parts, tools
\usepackage{hyperref}
\usepackage[toc,page]{appendix}
\usepackage{enumitem}

% Tables
\usepackage{tabularray}
\usepackage{pifont}
\usepackage{xcolor}
\usepackage{breqn}
\usepackage{booktabs}
\newcommand{\cmark}{\textcolor{green!80!black}{\ding{51}}}
\newcommand{\xmark}{\textcolor{red}{\ding{55}}}
\usepackage{multirow}

% For images
\usepackage{graphicx}
\usepackage[skip=0.5\baselineskip]{caption}
\usepackage{subcaption}
\usepackage{wrapfig}
\usepackage{tikz}
\usepackage{float} % For forcing placement

% Maths
\usepackage{amsmath}
\usepackage{amsfonts}
\usepackage{amsthm}
\usepackage{amssymb}
\usepackage{mathtools}
\usepackage{bbm}

\newtheorem{theorem}{Theorem}[section]
\newtheorem{proposition}{Proposition}[theorem]
\newtheorem{corollary}{Corollary}[theorem]
\newtheorem{lemma}[theorem]{Lemma}


\hypersetup{
colorlinks=true,
linkcolor=purple,
citecolor=purple,
filecolor=magenta,      
urlcolor=cyan,
pdfpagemode=FullScreen,
}



% Citing code
\usepackage{listings}

\renewcommand{\ttdefault}{cmtt}
\DeclareFixedFont{\ttb}{T1}{cmtt}{bx}{n}{12} % for bold
\DeclareFixedFont{\ttm}{T1}{cmtt}{m}{n}{12}  % for normal
% Custom colors
\usepackage{color}
\definecolor{deepblue}{rgb}{0,0,0.5}
\definecolor{deepred}{rgb}{0.6,0,0}
\definecolor{deepgreen}{rgb}{0,0.5,0}
% Python style for highlighting
\newcommand\pythonstyle{\lstset{
language=Python,
basicstyle=\ttm,
morekeywords={self},              % Add keywords here
keywordstyle=\ttb\color{deepblue},
emph={MyClass,__init__},          % Custom highlighting
emphstyle=\ttb\color{deepred},    % Custom highlighting style
stringstyle=\color{deepgreen},
frame=tb,                         % Any extra options here
showstringspaces=false
}}
\newcommand\pythoninline[1]{{\pythonstyle\lstinline!#1!}}
%====================== PACKAGES ======================

\usepackage[english]{babel}
\usepackage[utf8x]{inputenc}

% General document layout
\usepackage{abstract}
\usepackage[style=verbose-ibid,backend=bibtex]{biblatex}
\usepackage{hyperref}
\usepackage{tcolorbox}
\usepackage{setspace}
% Margins
\usepackage[T1]{fontenc}
\usepackage[top=2cm, bottom=2cm, left=2cm, right=2cm]{geometry}

% Scientific writing
\usepackage{amsmath}
\usepackage{amssymb}
\usepackage{bbm}
\usepackage{mathtools}

% Tikz illustrations
\usepackage{graphicx}
\usepackage{float}
\usepackage{tikz}
\usepackage{pgfplots}
\pgfplotsset{compat=1.18}

% Bibliography stuff
\usepackage[
backend=biber,
style=alphabetic,
sorting=ynt,
natbib=true
]{biblatex}

\addbibresource{biblio.bib}

%==== INFORMATION ET REGLES ======================

% Numbering
\setcounter{secnumdepth}{3}
\setcounter{tocdepth}{3}

\hypersetup{
pdfauthor = {Vivien Tisserand},
pdftitle = {Quantitative finance\\ Interviews preparation \& a bunch of other things},
}

%======================== DOCUMENT ========================

\begin{document}

% Line spacing
\newcommand{\HRule}{\rule{\linewidth}{0.5mm}}

\begin{titlepage}
   \begin{center}
       \vspace*{1cm}

        \Large
       \textbf{Quantitative finance}

        \large
       \vspace{0.5cm}
        Interviews preparation
            
       \vspace{1.5cm}

       \textbf{Vivien Tisserand}

       \vfill
            
       \vspace{0.8cm}
     

    \textbf{Abstract}
    \end{center}
    
    This is a summary of interview questions that I found and digressed on in quantitative finance. It is a mixed of applied mathematics and computer science.\\
    I am not found of brainteasers, they are a poor way to assess for a candidate's hability to be an asset for the teams. This work smoothly transitioned to a sort of \textit{vademecum} in applied mathematics : through several questions, it goes through several techniques that are easy to forget with time. I myself refer to it quite often when I forget about the way to solve an arithmetico-geometric sequence or the general solution of a second-order differential equation.
    
\end{titlepage}


\tableofcontents
\thispagestyle{empty}
\setcounter{page}{0}

\chapter{Probability}

\input{chapters/01__probability/coupons_collector.tex}
\section{Correlated bivariate distribution}

\begin{tcolorbox}[width=\linewidth, sharp corners=all, colback=white!95!black]
Let $(X,Y)$ follow a bivariate normal standard distribution with correlation $\rho$. Find the expectation: $$\mathbb{E}[\operatorname{sgn}(X)\operatorname{sgn}(Y)].$$
\end{tcolorbox}

$$(X,Y) \sim \mathcal{N}\left(\begin{pmatrix}0\\0\end{pmatrix}, \begin{pmatrix}
1 & \rho\\ \rho & 1
\end{pmatrix}\right).$$

To see what happens here, we can compare the density contour of this distribution with the independent case. The covariance matrix is symmetric thus diagonalizable. We can find its eigenvalues and its eigenvectors (through classic computations or noticing this is a circulant matrix). With $P = \begin{pmatrix}
1 & 1\\ 1 & -1
\end{pmatrix}$,
$$\Sigma = P\begin{pmatrix}
1+\rho & 0\\ 0 & 1-\rho
\end{pmatrix}P^{-1}.$$
This gives us the shape of the correlated distribution. Qualitatively, we can say that, as $\rho$ defines how rotated and squished the distribution is, the bigger $\rho$, the higher the probabiliy of $X$ and $Y$ being the same sign.

\begin{figure}[H]
\includegraphics[width=0.35\textwidth]{images/tikz/bivariate_contour.pdf}
\centering
\caption{Density contours of a bivariate normal law, with $\rho=0.7$}
\label{fig:bivariate_contour}
\end{figure}


Back to our problem: the random variable $\operatorname{sgn}(X)\operatorname{sgn}(Y)$ takes values in the set $\{-1,1\}$. Thus, to get its expectancy, we can compute these discrete probabilities :
\begin{align*}
    \mathbb{E}[\operatorname{sgn}(X)\operatorname{sgn}(Y)] &= 1\times \mathbb{P}(\operatorname{sgn}(X)\operatorname{sgn}(Y)=1) -1\times \mathbb{P}(\operatorname{sgn}(X)\operatorname{sgn}(Y)=-1)\\
    &= 1\times \mathbb{P}(\operatorname{sgn}(X)\operatorname{sgn}(Y)=1) -1\times (1-\mathbb{P}(\operatorname{sgn}(X)\operatorname{sgn}(Y)=1))\\
    &= 2\mathbb{P}(\operatorname{sgn}(X)\operatorname{sgn}(Y)=1) -1.
\end{align*}

Using the symmetry of the distribution, $$\mathbb{P}(\operatorname{sgn}(X)\operatorname{sgn}(Y)=1) = \mathbb{P}(X>0,Y>0) + \mathbb{P}(X<0,Y<0) = 2\mathbb{P}(X>0,Y>0),$$

thus the only thing we need to compute is $\mathbb{P}(X>0,Y>0)$.\\

If $$\begin{pmatrix}U\\V\end{pmatrix} = \Sigma^{-1/2}\begin{pmatrix}X\\Y\end{pmatrix},$$
then $(U,V)$ follows an independent bivariate normal standard distribution. Inverting $\Sigma$ we get:
$$\Sigma^{-1} = \frac1{1-\rho^2}\begin{pmatrix}
1 & -\rho\\ -\rho & 1
\end{pmatrix}.$$

\begin{figure}[H]
    \centering
    \includegraphics[width=0.8\textwidth]{images/corr_biv_normal.png}
    \caption{Area of the event $X>0,Y>0$ for $\rho = 0$ (left) and $\rho \ne 0$ (right). From \href{https://math.stackexchange.com/questions/1687795/correlated-joint-normal-distribution-calculating-a-probability}{here}.}
\end{figure}

Then, there exists a $\theta \in [0,2\pi]$ such that $\mathbb{P}(X>0,Y>0) = \dfrac{\theta}{2\pi}$. This $\theta$ verifies $$\cos{\theta} = \dfrac{\langle u,v\rangle}{\|u\| \|v\|}.$$ with $u = \Sigma^{-1/2} \begin{pmatrix}1\\0\end{pmatrix}$ and $v = \Sigma^{-1/2} \begin{pmatrix}0\\1\end{pmatrix}$


$$\langle u,v\rangle = (1\ 0)\,\Sigma^{-1}(0\ 1)^T=-\rho/(1-\rho^2)$$
$$\|u\|^2=(1\ 0)\,\Sigma^{-1}(1\ 0)^T=1/(1-\rho^2)$$
$$\|v\|^2=(0\ 1)\,\Sigma^{-1}(0\ 1)^T=1/(1-\rho^2)$$

so that $\cos(\theta)=-\rho.$ Putting it all together gives 
$$\mathbb{P}(X>0,Y>0)=\dfrac{\arccos(-\rho)}{ 2\pi}.$$

Finally, $$\fbox{\mathbb{E}[\operatorname{sgn}(X)\operatorname{sgn}(Y)] = \dfrac{2\arccos(-\rho)}{\pi} - 1.}$$

Note that if $\rho=0$, we have $\mathbb{E}[\operatorname{sgn}(X)\operatorname{sgn}(Y)]=0$ ; it converges to $1$ as $\rho \longrightarrow 1$ and to $-1$ as $\rho \longrightarrow -1$, which gives us confidence in our answer.


\chapter{Statistics}

\section{Estimating the support of an uniform law}

\begin{tcolorbox}[width=\linewidth, sharp corners=all, colback=white!95!black]
Suppose that we have $x_1,\dots,x_n$ observations from an uniform law $X \sim \mathcal{U}[0,\theta]$, where $\theta$ is an unknown parameter that we want to estimate. Give at least two estimators for $\theta$ and compare them.
\end{tcolorbox}

\begin{itemize}

\item \textbf{Method of moments:}

Having a look at the first order moment, it appears that $\mathbb{E}[X] = \dfrac{\theta}{2}$. Taking the empirical counter-party of this theoretical quantity, we have $\hat{\theta}^{\text{MM}} = \dfrac{2}{n}\sum\limits_{i=1}^n x_i$.\newline
By applying the strong law of large numbers and the continuous mapping theorem, $\hat{\theta}^{\text{MM}} \xrightarrow{a.s.} \theta$. Thus this estimator is consistent.\newline
We want asymptotic results on the convergence of this estimator. Before using the CLT, we have to check for the existence of a second-order moment.

\begin{align*}
    \mathbb{E}[X^2] &= \displaystyle \int_{\mathbb{R}} x^2 f(x) \, \mathrm{d}x \\
    &= \displaystyle \int_{0}^{\theta} x^2 \dfrac{1}{\theta} \, \mathrm{d}x \\
    &= \left[ \dfrac{1}{3\theta}x^3 \right]^\theta_0 \\
    &= \dfrac{\theta^2}{3} < +\infty
\end{align*}

Thus, we have $\mathbb{V}[X] = \mathbb{E}[X^2] - \mathbb{E}[X]^2 = \dfrac{\theta^2}{12}$. So, $\mathbb{V}[2X_1]  = \dfrac{\theta^2}{3}$.\newline

By applying the central limit theorem, we have : $$\sqrt{n}(\hat{\theta}^{\text{MM}} - \theta) \xrightarrow{(d)} \mathcal{N}\left(0, \dfrac{\theta^2}{3}\right).$$

We have to evaluate the risk of this estimator, that we write as the sum of the squared bias and the variance : $$\text{MSE}(\hat{\theta}^{\text{MM}}) = \mathbb{E}[(\hat{\theta}^{\text{MM}} - \theta)^2] = \mathbb{E}[(\hat{\theta}^{\text{MM}} - \mathbb{E}[\hat{\theta}^{\text{MM}}])^2] + \mathbb{E}[\hat{\theta}^{\text{MM}} - \theta]^2 = \mathbb{V}[\hat{\theta}^{\text{MM}}] + (\mathbb{E}[\hat{\theta}^{\text{MM}}] - \theta)^2.$$

We have $\mathbb{E}[\hat{\theta}^{\text{MM}}] = 0$ and $\mathbb{V}[\hat{\theta}^{\text{MM}}] = \dfrac{1}{n^2}n\mathbb{V}[2X_1] = \dfrac{\theta^2}{3n}.$\newline

Thus, $$\text{MSE}(\hat{\theta}^{\text{MM}}) = \dfrac{\theta^2}{3n}.$$


\item \textbf{Maximum likelihood:}

Let's write the likelihood of this model:

\begin{align*}
    L((X_1,\dots,X_n), \theta) &= \displaystyle \prod_{i=1}^{n} f_X(X_i) \\
    &= \displaystyle \prod_{i=1}^{n} \dfrac{1}{\theta} \mathbbm{1}_{[0, \theta]}(X_i)\\
    &= \dfrac{1}{\theta^n} \prod_{i=1}^{n} \mathbbm{1}_{[0, \theta]}(X_i).
\end{align*}

And this function is maximized by choosing the smallest $\theta$ such that all of the $X_i$ lie in $[0, \theta]$, that is $\hat{\theta}^{\text{MLE}} = \max_{1\leq i \leq n} X_i.$\newline

To check the consistency of this estimator, we will have a look at its convergence (in probability). Let $\theta \in \Theta$ and $\varepsilon >0$ :

\begin{align*}
    \mathbb{P}_\theta(\lvert\hat{\theta}^{\text{MLE}} - \theta\rvert \ge \varepsilon) &= \mathbb{P}_\theta(\hat{\theta}^{\text{MLE}} \ge \theta + \varepsilon) + \mathbb{P}_\theta(\hat{\theta}^{\text{MLE}} \le \theta - \varepsilon)\\
    &= 0 + \mathbb{P}_\theta(\max_{1\leq i \leq n} X_i \le \theta - \varepsilon)\\
    &= \prod_{i=1}^{n} \mathbb{P}_\theta(X_i \le \theta - \varepsilon)\\
    &= \left(1 - \dfrac{\varepsilon}{\theta} \right)^n \underset{n\to +\infty}{\longrightarrow} 0.
\end{align*}


Thus, $\hat{\theta}^{\text{MLE}} \xrightarrow{\mathbb{P}} \theta$ : this estimator is consistent.\newline

In order to estimate the risk of this estimator, we have to look at the law that the maximum of $n$ independent uniform laws follows. This is done by looking at the cumulative distribution function. Let $x \in [0,\theta] :$

\begin{align*}
    \mathbb{P}_\theta(X_{(n)} \le x) &= \mathbb{P}_\theta\left(\bigcap_{i=1}^{n} {X_i \le x}\right)\\
    &= \prod_{i=1}^{n} \mathbb{P}_\theta(X_i \le x)\\
    &= \left(\dfrac{x}{\theta}\right)^n.
\end{align*}

Thus, $$F_{X_{(n)}} = \begin{cases}
        0 & \text{if } x < 0\\
        \left(\dfrac{x}{\theta}\right)^n & \text{if } 0 \le x \le \theta\\
        1 & \text{if } x>\theta
        \end{cases}$$

This cdf as smooth as we need to take its derivative: that will be the density we were looking for:

$$f_{X_{(n)}}(x) = n \dfrac{x^{n-1}}{\theta^n} \mathbbm{1}_{[0, \theta]}(x)$$

Let's compute the bias and the variance.\newline

$$\mathbb{E}[\hat{\theta}^{\text{MLE}}] = \displaystyle \int_{\mathbb{R}} x f_{X_{(n)}}(x) \, \mathrm{d}x = \displaystyle \int_{0}^{\theta} \dfrac{n}{\theta^n} x^n \, \mathrm{d}x = \dfrac{n}{\theta^n} \left[ \dfrac{x^{n+1}}{n+1}\right]^\theta_0 = \dfrac{n}{n+1}\theta.$$

Then, the bias is : $B(\hat{\theta}^{\text{MLE}}) = \dfrac{n}{n+1}\theta - \theta = -\dfrac{1}{n+1}\theta \ne 0.$ We can introduce a corrected estimator that we will consider too : $\hat{\theta}_{\text{corr}}^{\text{MLE}} = \dfrac{n+1}{n} \hat{\theta}^{\text{MLE}}$, such that $\mathbb{E}[\hat{\theta}_{\text{corr}}^{\text{MLE}}] = \theta$:  an unbiased estimator.\newline

Then, we have $$\mathbb{E}[(\hat{\theta}^{\text{MLE}})^2] = \displaystyle \int_{\mathbb{R}} x^2 f_{X_{(n)}}(x) \, \mathrm{d}x = \displaystyle \int_{0}^{\theta} \dfrac{n}{\theta^n} x^{n+1} \, \mathrm{d}x = \dfrac{n}{\theta^n} \left[ \dfrac{x^{n+2}}{n+2}\right]^\theta_0 = \dfrac{n}{n+2}\theta^2.$$ And $$\text{MSE}(\hat{\theta}^{\text{MLE}}) = \mathbb{E}[(\hat{\theta}^{\text{MLE}} - \theta)^2] = \mathbb{E}[(\hat{\theta}^{\text{MLE}})^2] -2 \theta \mathbb{E}[\hat{\theta}^{\text{MLE}}] + \theta^2$$

Thus, $$\text{MSE}(\hat{\theta}^{\text{MLE}}) = \dfrac{n}{n+2}\theta^2 - 2\dfrac{n}{n+1}\theta^2 + \theta^2 = \dfrac{2\theta^2}{(n+1)(n+2)}.$$

And $$\text{MSE}(\hat{\theta}_{\text{corr}}^{\text{MLE}}) = \left(\dfrac{n+1}{n}\right)^2 \mathbb{E}[(\hat{\theta}^{\text{MLE}})^2] - 2\dfrac{n+1}{n}\theta \mathbb{E}[(\hat{\theta}^{\text{MLE}} + \theta^2 = \dfrac{\theta^2}{n(n+1)}.$$

\item \textbf{Maximum a posteriori:}

We write the likelihood of the model in terms of $\theta$ :
$$L((X_1,\dots,X_n), \theta) = \dfrac{1}{\theta^n} \prod_{i=1}^{n} \mathbbm{1}_{[0, \theta]}(X_i) = \dfrac{1}{\theta^n} \mathbbm{1}_{[X_{(n)}, =\infty[}(\theta).$$

\textbf{Remark :} the set such that $L(., \theta)>0$ is $[0, \theta]$ : it depends on $\theta$, thus the model is not regular. Keep that in mind when dealing with Fisher information for instance.\newline

\begin{enumerate}
    \item \textbf{Flat prior:}\newline

We apply the definition for a Bayesian estimator with a prior density $\pi_0$ :

\begin{align*}
    \hat{\theta}^{\text{B}} &= \dfrac{\displaystyle \int_{\Theta} \theta L(x, \theta) \pi_0(\theta) \, \mathrm{d}\lambda(\theta)}{\displaystyle \int_{\Theta} L(x, \theta) \pi_0(\theta) \, \mathrm{d}\lambda(\theta)}\\
    &= \dfrac{\displaystyle \int_{X_{(n)}}^{+\infty} \theta^{-n+1} \, \mathrm{d}\theta}{\displaystyle \int_{X_{(n)}}^{+\infty} \theta^{-n} \, \mathrm{d}\theta}\\
    &= \dfrac{n-1}{n-2}X_{(n)}.
\end{align*}

Bias and MSE are not computed there for sanity reasons.

    \item \textbf{Jeffreys prior:}\newline

The density function of this prior is proportional to the squareroot of the determinant of the Fisher information matrix.\newline

Thus we need to compute this quantity for this model, with n observations (as it is not regular, $I_n \ne n I_1$) :

$$I_n(\theta) = \mathbb{E}\left[\dfrac{\partial \log L_n(\theta)}{\partial \theta}^2\right]$$

We have $I_n(\theta) = \mathbb{E}[(-n/\theta)^2] = \dfrac{n^2}{\theta^2}$\newline

(If we had taken the expectancy of the second-order derivative of the log-likelihood, we would not have had the same result has the model is not regular.)\newline

This gives us the noninformative prior (Jeffreys) : $\pi_0(\theta) \propto \theta^{-1}.$


\end{enumerate}



\end{itemize}




\chapter{Machine learning}

\section{Linear regression}

\begin{tcolorbox}[width=\linewidth, sharp corners=all, colback=white!95!black]
We are interested in the basic linear regression model where given observations $(x_i, y_i)_{1\leq i \leq n}$ we want to build the model $$Y = \alpha + \beta X + \varepsilon.$$

Explain the underlying assumptions in the model and derive estimators for the coefficients.

\end{tcolorbox}

The underlying assumptions of the linear model are the following:

\begin{itemize}
    \item No perfect multicollinearity between the explanatory variables, otherwise the parameter $\beta$ is not identifiable. This is the \textbf{full-rank} assumption.
    \item Independence of errors. Generalized least squares can handle correlated errors.
    \item \textbf{Homoscedasticity}, or constant variance, which can be tested on the residuals. If there is heteroscedasticity, the Gauss-Markov theorem doesn't apply, thus the estimators derived are not the Best Linear Unbiased Estimators (BLUE). It can be corrected thanks to a weighted least squares approach, or a logarithmization of the data.
    \item \textbf{Exogeneity}.
\end{itemize}

We have to minimize the Euclidean distance between the predicted values by the model for $Y$, $\hat{Y}$ and the actual values. This can be written as :
$$(\hat{\alpha}, \hat{\beta}) \in \operatorname{argmin}_{(\alpha, \beta) \in \mathbb{R}^2} \displaystyle \sum_{i=1}^{n} (y_i - \hat{y_i})^2$$

Replacing with the model, we have to find parameters that minimize $f(\alpha, \beta) = \displaystyle \sum_{i=1}^{n} (y_i - (\alpha + \beta x_i))^2.$ We will write the first order conditions and check by computing the Hessian that we are actually looking at a minimum.

$$\left\{\begin{array}{ll}
        \dfrac{\partial f}{\partial \alpha}(\alpha, \beta) &= 0 \\
        \dfrac{\partial f}{\partial \beta}(\alpha, \beta) &= 0
    \end{array}\right. \Leftrightarrow 
\left\{\begin{array}{ll}
        -2\sum_i (y_i - \alpha - \beta x_i) &= 0 \\
        -2\sum_i x_i (y_i - \alpha - \beta x_i) &= 0
    \end{array}\right.$$

The first line give $\hat{\alpha} = \overline{y} - \hat{\beta} \overline{x}$ while the second line can be written as the following : 
\begin{equation*}
\begin{aligned}
    \sum_i x_i (y_i - \hat{\alpha} - \hat{\beta} x_i) &= \sum_i y_i x_i - (\overline{y} - \hat{\beta} \overline{x})x_i - \hat{\beta} x_i^2\\
    &= \sum_i y_i - \overline{y} + \hat{\beta} (\overline{x} - x_i).
\end{aligned}
\end{equation*}

Thus 
\begin{equation*}
\begin{aligned}
    &\sum_i x_i (y_i - \hat{\alpha} - \hat{\beta} x_i) = 0 &\Leftrightarrow \sum_i y_i - \overline{y} + \hat{\beta} (\overline{x} - x_i) = 0\\
    &\Leftrightarrow \hat{\beta} = \dfrac{\sum_i y_i - \overline{y}}{\sum_i x_i - \overline{x}}\\
    &\Leftrightarrow \hat{\beta} = \dfrac{\sum_i (y_i - \overline{y})(x_i-\overline{x})}{\sum_i (x_i - \overline{x})^2}\\
    &\Leftrightarrow \boxed{\hat{\beta} = \dfrac{\widehat{\operatorname{Cov}}(X, Y)}{\widehat{\operatorname{Var}}(X)}.}
\end{aligned}
\end{equation*}





\chapter{Stochastic calculus}

\section{Bayes classifier}

\begin{tcolorbox}[width=\linewidth, sharp corners=all, colback=white!95!black]
    Considers a Brownian diffusion process $(X_t)_{t\in [0,T]}$ solution of the following stochastic differential equation (SDE):
    \begin{equation*}\label{SDE}
    dX_t=b(t, X_t)dt+\sigma(t,X_t)dW_t,\qquad X_0=x.
    \end{equation*}
    where $b:[0,T]\times \mathbb{R} \to \mathbb{R}$, $\sigma :[0,T]\times\mathbb{R}\to \mathbb{R}$ are continuous functions and $(W_t)_{t\in [0,T]}$ denotes a Brownian motion defined on a filtered probability space.\newline

    Under which conditions is this diffusion recurrent? Positive recurrent?\\

    Application: discuss the recurrent or the Ornstein-Uhlenbeck process:
    \begin{equation*}\label{OU}
        dX_t=-\mu(X_t - m) dt+\sigmadW_t.
    \end{equation*}

\end{tcolorbox}

\chapter{Finance}

\section{Betting on volatility -- around varswaps}

\begin{tcolorbox}[width=\linewidth, sharp corners=all, colback=white!95!black]
    A desk is interested in expressing a view on volatility. To do so they suggest entering a variance swap contract with payoff:
    \[\dfrac{1}{T} \displaystyle \int_{0}^{T} \sigma_t^2 dt - \sigma_K^2\]

    On the sell side, how would you price and hedge such a contract?\newline
    On the buy side, if your goal is to tail-hedge your portfolio, could you suggest other approaches than buying a varswap?

\end{tcolorbox}
\input{chapters/05__finance/dupire_formula.tex}



\newpage

\printbibliography[heading=bibintoc,title={References}]

\end{document}


